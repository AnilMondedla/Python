\documentclass{beamer}
\mode<presentation>
\usepackage{amsmath}
\usepackage{amssymb}
%\usepackage{advdate}
\usepackage{adjustbox}
\usepackage{subcaption}
\usepackage{enumitem}
\usepackage{multicol}
\usepackage{mathtools}
\usepackage{listings}
\usepackage{url}
\def\UrlBreaks{\do\/\do-}
\usetheme{Boadilla}
\usecolortheme{lily}
\setbeamertemplate{footline}
{
  \leavevmode%
  \hbox{%
  \begin{beamercolorbox}[wd=\paperwidth,ht=2.25ex,dp=1ex,right]{author in head/foot}%
    \insertframenumber{} / \inserttotalframenumber\hspace*{2ex} 
  \end{beamercolorbox}}%
  \vskip0pt%
}
\setbeamertemplate{navigation symbols}{}

\providecommand{\nCr}[2]{\,^{#1}C_{#2}} % nCr
\providecommand{\nPr}[2]{\,^{#1}P_{#2}} % nPr
\providecommand{\mbf}{\mathbf}
\providecommand{\pr}[1]{\ensuremath{\Pr\left(#1\right)}}
\providecommand{\qfunc}[1]{\ensuremath{Q\left(#1\right)}}
\providecommand{\sbrak}[1]{\ensuremath{{}\left[#1\right]}}
\providecommand{\lsbrak}[1]{\ensuremath{{}\left[#1\right.}}
\providecommand{\rsbrak}[1]{\ensuremath{{}\left.#1\right]}}
\providecommand{\brak}[1]{\ensuremath{\left(#1\right)}}
\providecommand{\lbrak}[1]{\ensuremath{\left(#1\right.}}
\providecommand{\rbrak}[1]{\ensuremath{\left.#1\right)}}
\providecommand{\cbrak}[1]{\ensuremath{\left\{#1\right\}}}
\providecommand{\lcbrak}[1]{\ensuremath{\left\{#1\right.}}
\providecommand{\rcbrak}[1]{\ensuremath{\left.#1\right\}}}
\theoremstyle{remark}
\newtheorem{rem}{Remark}
\newcommand{\sgn}{\mathop{\mathrm{sgn}}}
\providecommand{\abs}[1]{\left\vert#1\right\vert}
\providecommand{\res}[1]{\Res\displaylimits_{#1}} 
\providecommand{\norm}[1]{\lVert#1\rVert}
\providecommand{\mtx}[1]{\mathbf{#1}}
\providecommand{\mean}[1]{E\left[ #1 \right]}
\providecommand{\fourier}{\overset{\mathcal{F}}{ \rightleftharpoons}}
%\providecommand{\hilbert}{\overset{\mathcal{H}}{ \rightleftharpoons}}
\providecommand{\system}{\overset{\mathcal{H}}{ \longleftrightarrow}}
	%\newcommand{\solution}[2]{\textbf{Solution:}{#1}}
%\newcommand{\solution}{\noindent \textbf{Solution: }}
\providecommand{\dec}[2]{\ensuremath{\overset{#1}{\underset{#2}{\gtrless}}}}
\newcommand{\myvec}[1]{\ensuremath{\begin{pmatrix}#1\end{pmatrix}}}
\let\vec\mathbf

\lstset{
%language=C,
frame=single, 
breaklines=true,
columns=fullflexible
}

\numberwithin{equation}{section}

\title{Presentation}
\author{Mondedla Anil \\ Dept. of Electrical Engg.,\\ Assignment 6}

\date{\today} 
\begin{document}

\begin{frame}
\titlepage
\end{frame}

\section*{Outline}
\begin{frame}
\tableofcontents
\end{frame}
\section{Problem}
\begin{frame}
\frametitle{Problem Statement}
Prove that the points (2,-2),(8,4),(5,7), and (-1,1) are at the angular points of a rectangle. 
\end{frame}

%\subsection{Literature}
\section{Solution}
\subsection{Vector Representation}
\begin{frame}
\frametitle{Vector Representation}
\begin{align}
\vec{A} = \myvec{2\\-2 },
\vec{B} = \myvec{8\\4 },
\vec{C} = \myvec{5\\7},
\vec{D} = \myvec{-1\\1}
\end{align}
\end{frame}
\subsection{Direction Vector}
\begin{frame}
\frametitle{Direction Vector}
Direction vector of A and B 
\begin{align}
\vec{m}_1= \myvec{2\\-2 }-\myvec{8\\4 } = \myvec{-6\\-6 }
\end{align}
Direction vector of B and C 
\begin{align}
\vec{m}_2= \myvec{8\\4 }-\myvec{5\\7 } = \myvec{3\\-3 }
\end{align}
Direction vector of C and D 
\begin{align}
\vec{m}_3= \myvec{5\\7 }-\myvec{-1\\1 } = \myvec{6\\6 }
\end{align}
Direction vector of D and A 
\begin{align}
\vec{m}_4= \myvec{-1\\1 }-\myvec{2\\-2 } = \myvec{-3\\3 }
\end{align}
\end{frame}

\begin{frame}[fragile]
\begin{align}
     \text{These direction vectors are parallel}\nonumber\\
     \vec{m}_1=k\vec{m}_3 \\
    \text{These direction vectors are parallel}\nonumber\\
    \vec{m}_2=k\vec{m}_4
\end{align}
\end{frame}

\subsection{For a Rectangle}
\begin{frame}
\frametitle{For a Rectangle}
\begin{align}
    (\vec{m}_1)^\top\vec{m}_2=0\\
    \myvec{-6\\-6}^\top\myvec{3\\-3}=0\\
    \myvec{-6&-6}\myvec{3\\-3}=0\\
    -18+18=0 \nonumber \\
    0=0 \nonumber
\end{align}
Hence, given points A,B,C and D are the Vector points of a rectangle.
\end{frame}

\begin{frame}{Plot}
\subsection{Plot}
\begin{figure}[H]
\centering
\includegraphics[width=0.8\columnwidth]{image.png}
\caption{}
\label{fig:1}
\end{figure}  
\end{frame}
\end{document}